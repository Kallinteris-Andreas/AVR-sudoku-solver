\documentclass[12pt, a4]{article}

\usepackage{ucs}
\usepackage[utf8x]{inputenc}
\usepackage[greek, english]{babel}

\usepackage{graphicx}
\graphicspath{ {./images/} }

\usepackage{geometry}

\begin{document}

\newgeometry{left=0.8in,right=0.8in,top=1in,bottom=1in}
\noindent \rule{\textwidth}{3pt}
\begin{center}
	{\bf \Large{Milestone 1 Report}}\\
	ACE411-Embedded Microprocessor Systems \\
	Winter semester of academic year 2021-2022
	\rule{\textwidth}{0.2mm} 
	\begin{tabular}{l r}
		Kallinteris Andreas:& 2017030066 \\ 
		Lioudakis Emmanouil:      &2018030020 
	\end{tabular} \vrule \hspace{3mm}
	\indent	Team number on eclass:  5
	\rule{\textwidth}{1pt}
\end{center}

\section*{STK500 configuration}

\section*{Microchip configuration - Compilation process}
Since the code is written using the C++ language (using newer standards than the supported from Microchip Studio), some modifications should be done to the default settings of a new C++ project.  \\
Firstly, add the symbol “AVR” to the compiler’s symbols:\\
IMAGE 1\\
Secondly, add the flag “-std=c++17” to the compiler’s flags:\\
IMAGE 2\\
The optimization level should be set to “-O2 (Optimize more).\\
Finally, the SIMULATION\_MODE should be defined (by uncommenting the line 25 of main.cpp). When it is defined, the program will read from TCNT2 instead of UDR and will redi-rect its output from UDR to TCNT0. By doing that, the program can be simulated with the provided stimuli files.

\section*{Description of the program}

\subsubsection*{The sudoku solving algorithm}
TO BE COMPLETED BY ANDREAS

\subsubsection*{Controlling the LED progress bar}
by elioudakis

\subsubsection*{UART}
by elioudakis

\section*{Resource usage}
\subsubsection*{Program memory (flash)}

\subsubsection*{Static RAM (SRAM)}

\section*{Simulation in Microchip Studio, using stimuli files}
X stimuli files are submitted with the code:
\begin{itemize}
\item	a, which feeds the program with one sudoku board (the one shown in the assign-ment), waits until the sudoku is solved and sends the results back.
\item	a, which feeds the program with the same sudoku board as above, but while solving, a “break” command stops the solving process, and using the “debug” command, the contents of some cells are read.
\item	a, which feeds the program with two sudoku boards, one after another. After solving the first and sending back the results, a “clear” command is executed and then the grid is filled with the second sudoku, it is solved and the results are sent back to the serial port.
\end{itemize}

\section*{Testing the code on real hardware (STK500)}
\subsubsection*{Using PuTTY}
\subsubsection*{Using the interface program developed by Odysseas Stavrou}


\end{document}